%
% FH Technikum Wien
% !TEX encoding = UTF-8 Unicode
%
% Erstellung von Master- und Bachelorarbeiten an der FH Technikum Wien mit Hilfe von LaTeX und der Klasse TWBOOK
%
% Um ein eigenes Dokument zu erstellen, müssen Sie folgendes ergänzen:
% 1) Mit \documentclass[..] einstellen: Master- oder Bachelorarbeit, Studiengang und Sprache
% 2) Mit \newcommand{\FHTWCitationType}.. Zitierstandard festlegen (wird in der Regel vom Studiengang vorgegeben - bitte erfragen)
% 3) Deckblatt, Kurzfassung, etc. ausfüllen
% 4) und die Arbeit schreiben (die verwendeten Literaturquellen in Literatur.bib eintragen)
%
% Getestet mit TeXstudio mit Zeichenkodierung ISO-8859-1 (=ansinew/latin1) und MikTex unter Windows
% Zu beachten ist, dass die Kodierung der Datei mit der Kodierung des paketes inputenc zusammen passt!
% Die Kodierung der Datei twbook.cls MUSS ANSI betragen!
% Bei der Verwendung von UTF8 muss dnicht nur die Kodierung des Dokuments auf UTF8 gestellt sein, sondern auch die des BibTex-Files!
%
% Bugreports und Feedback bitte per E-Mail an latex@technikum-wien.at
%
% Versionen
% *) V0.7: 9.1.2015, RO: Modeline angepasst und verschoben
% *) V0.6: 10.10.2014, RO: Weitere Anpassung an die UK
% *) V0.5: 8.8.2014, WK: Literaturquellen überarbeitet und angepasst
% *) V0.4: 4.8.2014, WK: Initalversion in SVN eingespielt
%
\documentclass[MMR,Master,nenglish]{twbook}%\documentclass[Bachelor,BMR,ngerman]{twbook}
\usepackage[utf8]{inputenc}
\usepackage[T1]{fontenc}

%
% Bitte in der folgenden Zeile den Zitierstandard festlegen
\newcommand{\FHTWCitationType}{IEEE} % IEEE oder HARVARD möglich - wenn Sie zwischen IEEE und HARVARD wechseln, bitte die temorären Dateien (aux, bbl, ...) löschen
%
\ifthenelse{\equal{\FHTWCitationType}{HARVARD}}{\usepackage{harvard}}{\usepackage{bibgerm}}

%
% Bei Bedarf bitte hier die Syntax-Highlightings anpassen
%
\usepackage[final]{listings}
\lstset{captionpos=b, numberbychapter=false,caption=\lstname,frame=single, numbers=left, stepnumber=1, numbersep=2pt, xleftmargin=15pt, framexleftmargin=15pt, numberstyle=\tiny, tabsize=3, columns=fixed, basicstyle={\fontfamily{pcr}\selectfont\footnotesize}, keywordstyle=\bfseries, commentstyle={\color[gray]{0.33}\itshape}, stringstyle=\color[gray]{0.25}, breaklines, breakatwhitespace, breakautoindent}
\lstloadlanguages{[ANSI]C, C++, [gnu]make, gnuplot, Matlab}

%Formatieren des Quellcodeverzeichnisses
\makeatletter
% Setzen der Bezeichnungen für das Quellcodeverzeichnis/Abkürzungsverzeichnis in Abhängigkeit von der eingestellten Sprache
\providecommand\listacroname{}
\@ifclasswith{twbook}{english}
{%
    \renewcommand\lstlistingname{Code}
    \renewcommand\lstlistlistingname{List of Code}
    \renewcommand\listacroname{List of Abbreviations}
}{%
    \renewcommand\lstlistingname{Quellcode}
    \renewcommand\lstlistlistingname{Quellcodeverzeichnis}
    \renewcommand\listacroname{Abkürzungsverzeichnis}
}
% Wenn die Option listof=entryprefix gewählt wurde, Definition des Entyprefixes für das Quellcodeverzeichnis. Definition des Macros listoflolentryname analog zu listoflofentryname und listoflotentryname der KOMA-Klasse
\@ifclasswith{scrbook}{listof=entryprefix}
{%
    \newcommand\listoflolentryname\lstlistingname
}{%
}
\makeatother
\newcommand{\listofcode}{\phantomsection\lstlistoflistings}

% Die nachfolgenden Pakete stellen sonst nicht benötigte Features zur Verfügung
\usepackage{blindtext}

%
% Einträge für Deckblatt, Kurzfassung, etc.
%
\title{Comparison between Microservice and Monolithic API in a Cloud Environment}
\author{Florian Feka}
\studentnumber{1910257104}
%\author{Titel Vorname Name, Titel\and{}Titel Vorname Name, Titel}
%\studentnumber{XXXXXXXXXXXXXXX\and{}XXXXXXXXXXXXXXX}
\supervisor{Marvin Kosmider, BSc.}
%\supervisor[Begutachter]{Titel Vorname Name, Titel}
%\supervisor[Begutachterin]{Titel Vorname Name, Titel}
%\secondsupervisor{Titel Vorname Name, Titel}
%\secondsupervisor[Begutachter]{Titel Vorname Name, Titel}
%\secondsupervisor[Begutachterinnen]{Titel Vorname Name, Titel}
\place{Vienna}
\kurzfassung{\blindtext}
\schlagworte{Schlagwort1, Schlagwort2, Schlagwort3, Schlagwort4}
\outline{\blindtext}
\keywords{Keyword1, Keyword2, Keyword3, Keyword4}
%\acknowledgements{\blindtext}

\begin{document}

%Festlegungen für den HARVARD-Zitierstandard
\ifthenelse{\equal{\FHTWCitationType}{HARVARD}}{
\bibliographystyle{Harvard_FHTW_MR}%Zitierstandard FH Technikum Wien, Studiengang Mechatronik/Robotik, Version 1.2e
\citationstyle{dcu}%Correct citation-style (Harvardand, ";" between citations, "," between author and year)
\citationmode{abbr}%use "et al." with first citation
\renewcommand{\harvardand}{\&}%Harvardand in Zitaten
%Englisch
\newcommand{\citepic}[1]{(Source: \protect\cite{#1})}%Zitat: Bild
\newcommand{\citefig}[2]{(Source: \protect\cite{#1}, p. #2)}%Zitat: Bild aus Dokument
\newcommand{\citefigm}[2]{(Source: taken with modification from \protect\cite{#1}, p. #2)}%Zitat: modifiziertes Bild aus Dokument
\newcommand{\citep}{\citeasnoun}%In-Line Zitiat entweder mit \citep{} oder \citeasnoun{}
\newcommand{\acessedthrough}{Available at:}%Für URL-Angabe
\newcommand{\acessedthroughp}{Available through:}%Für URL-Angabe (Geschützte Datenbank, Zugriff durch FH)
\newcommand{\acessedat}{Accessed}%Für URL-Datum-Angabe
\newcommand{\singlepage}{p.}%Für Seitenangabe (einzelne Seite)
\newcommand{\multiplepages}{pp.}%Für Seitenangabe (mehrere Seiten)
\newcommand{\chapternr}{Ch.}%Für Kapitelangabe
\newcommand{\abstractonly}{Abstract only}
\newcommand{\edition}{~edition}%Edition -> note, that you have to write "edition = {2nd},"!
\iflanguage{ngerman}{
    %Deutsch Neue Rechtschreibung
    \renewcommand{\citepic}[1]{(Quelle: \protect\cite{#1})}%Zitat: Bild
    \renewcommand{\citefig}[2]{(Quelle: \protect\cite{#1}, S. #2)}%Zitat: Bild aus Dokument
    \renewcommand{\citefigm}[2]{(Quelle: modifiziert "ubernommen aus \protect\cite{#1}, S. #2)}%Zitat: modifiziertes Bild aus Dokument
    \renewcommand{\citep}{\citeasnoun}%In-Line Zitiat entweder mit \citep{} oder \citeasnoun{}
    \renewcommand{\acessedthrough}{Verf{\"u}gbar unter:}%Für URL-Angabe
    \renewcommand{\acessedthroughp}{Verf{\"u}gbar bei:}%Für URL-Angabe (Geschützte Datenbank, Zugriff durch FH)
    \renewcommand{\acessedat}{Zugang am}%Für URL-Datum-Angabe
    \renewcommand{\singlepage}{S.}%Für Seitenangabe (einzelne Seite)
    \renewcommand{\multiplepages}{S.}%Für Seitenangabe (mehrere Seiten)
    \renewcommand{\chapternr}{K.}%Für Kapitelangabe
    \renewcommand{\abstractonly}{ausschließlich Abstract}
    \renewcommand{\edition}{. Auflage}%Angabe der Auflage
}{
\iflanguage{german}{
    %Deutsch
    \renewcommand{\citepic}[1]{(Quelle: \protect\cite{#1})}%Zitat: Bild
    \renewcommand{\citefig}[2]{(Quelle: \protect\cite{#1}, S. #2)}%Zitat: Bild aus Dokument
    \renewcommand{\citefigm}[2]{(Quelle: modifiziert "ubernommen aus \protect\cite{#1}, S. #2)}%Zitat: modifiziertes Bild aus Dokument
    \renewcommand{\citep}{\citeasnoun}%In-Line Zitiat entweder mit \citep{} oder \citeasnoun{}
    \renewcommand{\acessedthrough}{Verf{\"u}gbar unter:}%Für URL-Angabe
    \renewcommand{\acessedthroughp}{Verf{\"u}gbar bei:}%Für URL-Angabe (Geschützte Datenbank, Zugriff durch FH)
    \renewcommand{\acessedat}{Zugang am}%Für URL-Datum-Angabe
    \renewcommand{\singlepage}{S.}%Für Seitenangabe (einzelne Seite)
    \renewcommand{\multiplepages}{S.}%Für Seitenangabe (mehrere Seiten)
    \renewcommand{\chapternr}{K.}%Für Kapitelangabe
    \renewcommand{\abstractonly}{ausschließlich Abstract}
    \renewcommand{\edition}{. Auflage}%Angabe der Auflage
}{%
}}}

\maketitle

%
% .. und hier beginnt die eigentliche Arbeit. Viel Erfolg beim Verfassen!
%
\chapter{Introduction}
Companies nowadays have many options on how to provide their software product to their customers, and with this, many things to consider when choosing a hosting platform. Corporations can choose to deploy on-premise, handling everything from the software to maintaining servers or delegate some of that work to cloud providers. Applications can be deployed through Infrastructure as a Service (IaaS)\cite{microIaas} or platform as a Service (PaaS)\cite{redPaas} and have a chance to take advantage of other services provided by Cloud Providers, like auto-scaling, high availability, continuous delivery and more. Most companies that move an application to the Cloud predominantly move a monolithic application.
\\
\\
Companies move to the cloud in hope of utilising features of IaaS/PaaS solutions to improve efficiency in their processes and to improve scaling to brace for peaks in requests. Most software applications developed by companies are three-tiered web applications, commonly using Java, Net, and PHP. Many of those applications face problems when migrating to the cloud since many architecture styles commonly used when developing web applications do not consider the ability to add/remove servers on demand and do not consider the option for multiple server instances to run.
\\
\\
Monoliths are applications built in, often one, big code base. There are variations since some monoliths are split according to their overarching responsibility. For example, a monolith that has one code base including the back-end and front-end(e.g. Spring in combination with Thymeleaf) could be split into Spring being the back-end and Angular being the front-end.
\\
\\
Microservices are separate, small, modular services that can be independently developed, updated, deployed and managed. Each service runs in its process and communicates over the network using a well-defined communication protocol. These services are self-describing and can be discovered and used by other processes without human intervention. Microservices are loosely coupled and can be easily scaled up and down based on the current business needs. They can also be used to develop applications much faster than a monolithic approach because each service can be developed independently of one another\cite{ade2017}.

%
% Hier beginnen die Verzeichnisse.
%
\clearpage
\ifthenelse{\equal{\FHTWCitationType}{HARVARD}}{}{\bibliographystyle{gerabbrv}}
\bibliography{Literatur}
\clearpage

% Das Abbildungsverzeichnis
\listoffigures
\clearpage

% Das Tabellenverzeichnis
\listoftables
\clearpage

% Das Quellcodeverzeichnis
\listofcode
\clearpage

\phantomsection
\addcontentsline{toc}{chapter}{\listacroname}
\chapter*{\listacroname}
\begin{acronym}[XXXXX]
    \acro{ABC}[ABC]{Alphabet}
    \acro{WWW}[WWW]{world wide web}
    \acro{ROFL}[ROFL]{Rolling on floor laughing}
\end{acronym}

%
% Hier beginnt der Anhang.
%
\clearpage
\appendix
\chapter{Anhang A}
\clearpage
\chapter{Anhang B}
\end{document}
